\documentclass{article}

\usepackage{graphicx} % Required for inserting images
\usepackage[left=1in,right=1in,top=1in,bottom=1in]{geometry} \usepackage{amsmath}
\usepackage{amsthm} %proof environment
\usepackage{amsthm} %proof environment
\usepackage{amssymb}
\usepackage{amsfonts}
\usepackage{enumitem} %nice lists
\usepackage{verbatim} %useful for something 
\usepackage{xcolor}
\usepackage{setspace}
\usepackage{titlesec}
\usepackage{blindtext} % I have no idea what this is 
\usepackage{caption}  % need this for unnumbered captions/figures
\usepackage{natbib}
\usepackage{appendix}
\usepackage{tikz}
\usepackage{hyperref}


\hypersetup{
    colorlinks=true,
    linkcolor=blue,
    filecolor=magenta,      
    urlcolor=blue,
    pdftitle={Overleaf Example},
    pdfpagemode=FullScreen,
    }

\titleformat{\section}{\bfseries\Large}{\thesection. }{5pt}{}

\begin{document}

\title{Multiscale Expansion for Rotating Stratified Turbulence}
\author{Dante Buhl, Pascale Garaud, Gregory P. Chini}


\newcommand{\wrms}{w_{\text{rms}}}
\newcommand{\bs}[1]{\boldsymbol{#1}}
\newcommand{\tb}[1]{\textbf{#1}}
\newcommand{\bmp}[1]{\begin{minipage}{#1\textwidth}}
\newcommand{\emp}{\end{minipage}}
\newcommand{\R}{\mathbb{R}}
\newcommand{\C}{\mathbb{C}}
\newcommand{\N}{\mathcal{N}}
%\newcommand{\K}{\bs{\mathrm{K}}}
\newcommand{\m}{\bs{\mu}_*}
\newcommand{\s}{\bs{\Sigma}_*}
\newcommand{\dt}{\Delta t}
\newcommand{\dx}{\Delta x}
\newcommand{\tr}[1]{\text{Tr}(#1)}
\newcommand{\Tr}[1]{\text{Tr}(#1)}
\newcommand{\Div}{\nabla \cdot}
\renewcommand{\div}{\nabla \cdot}
\newcommand{\Curl}{\nabla \times}
\newcommand{\Grad}{\nabla}
\newcommand{\grad}{\nabla}
\newcommand{\gradhs}{\nabla_{\perp S}}
\newcommand{\gradhf}{\nabla_{\perp F}}
\newcommand{\grads}{\nabla_s}
\newcommand{\gradf}{\nabla_f}
\newcommand{\xs}{x_s}
\newcommand{\x}{\bs{x}}
\newcommand{\xf}{x_f}
\newcommand{\ts}{t_s}
\newcommand{\tf}{t_f}
\newcommand{\pt}{\partial t}
\newcommand{\pz}{\partial z}
\newcommand{\uvec}{\bs{u}}
\newcommand{\uvecp}{\bs{u}_{\perp}}
\newcommand{\baruvecp}{\bar{\bs{u}}_{\perp}}
\newcommand{\brauvecp}{\bracket{\bs{u}_{\perp}}}
\newcommand{\perpuvecp}{\bs{u}'_{\perp}}
\newcommand{\bvec}{\bs{B}}
\newcommand{\nvec}{\hat{\bs{n}}}
\newcommand{\tu}{\tilde{\uvec}}
\newcommand{\B}{\bs{B}}
\newcommand{\A}{\bs{A}}
\newcommand{\jvec}{\bs{j}}
\newcommand{\bracket}[1]{\left<#1\right>}
\newcommand{\F}{\bs{F}}
\newcommand{\T}{\tilde{T}}
\newcommand{\ez}{\bs{e}_z}
\newcommand{\ex}{\bs{e}_x}
\newcommand{\ey}{\bs{e}_y}
\newcommand{\eo}{\bs{e}_{\bs{\Omega}}}
\newcommand{\ppt}[1]{\frac{\partial #1}{\partial t}}
\newcommand{\pp}[2]{\frac{\partial #1}{\partial #2}}
\newcommand{\pptwo}[2]{\frac{\partial^2 #1}{\partial #2^2}}
\newcommand{\ddtwo}[2]{\frac{d^2 #1}{d #2^2}}
\newcommand{\DDt}[1]{\frac{D #1}{D t}}
\newcommand{\ppts}[1]{\frac{\partial #1}{\partial t_s}}
\newcommand{\pptf}[1]{\frac{\partial #1}{\partial t_f}}
\newcommand{\ppz}[1]{\frac{\partial #1}{\partial z}}
\newcommand{\ddz}[1]{\frac{d #1}{d z}}
\newcommand{\ppzetas}[1]{\frac{\partial^2 #1}{\partial \zeta^2}}
\newcommand{\ppzs}[1]{\frac{\partial #1}{\partial z_s}}
\newcommand{\ppzf}[1]{\frac{\partial #1}{\partial z_f}}
\newcommand{\ppx}[1]{\frac{\partial #1}{\partial x}}
\newcommand{\ddx}[1]{\frac{d #1}{d x}}
\newcommand{\ppxi}[1]{\frac{\partial #1}{\partial x_i}}
\newcommand{\ppxj}[1]{\frac{\partial #1}{\partial x_j}}
\newcommand{\ppy}[1]{\frac{\partial #1}{\partial y}}
\newcommand{\ppzeta}[1]{\frac{\partial #1}{\partial \zeta}}
\renewcommand{\k}{\bs{k}}
\newcommand{\real}[1]{\text{Re}\left[#1\right]}
\newcommand{\bexp}{\left(\bracket{b} + \bar{b} + b'\right)}
\newcommand{\upexp}{\left(\bracket{\uvec_{\perp}} + \bar{\uvec}_{\perp} + \uvec'_{\perp}\right)}
\newcommand{\uexp}{\left(\bracket{\uvec} + \bar{\uvec} + \uvec'\right)}
\newcommand{\wexp}{\left(\bracket{w} + \bar{w} + w'\right)}
\newcommand{\pexp}{\left(\bracket{p} + \bar{p} + p'\right)}


\maketitle 
% This line removes the automatic indentation on new paragraphs

\setlength{\parindent}{0pt}
\section{Equations, Component Expansion, and Scales}

\subsection{Governing Equations for Rotating Stratified Flows}
\begin{align}
    \partial_t \uvec + \uvec\cdot\grad\uvec + \frac{1}{Ro}\left(\ez \times
    \uvec\right) = - \grad p + \frac{1}{Fr^2}b\ez + \F +
    \frac{1}{Re}\grad^2\uvec \label{eq:NS}\\
    \partial_t b + \uvec\cdot\grad b + w = \frac{1}{Pe}\grad^2b \label{eq:buoy}\\
    \grad\cdot\uvec = 0 \label{eq:div}
\end{align}
Here we define our nondimensionalized equations along with the
nondimensional numbers: the Rossby $Ro$,
Reynolds $Re$, Froude $Fr$ and P\'eclet $Pe$ numbers defined according to the
following characteristic scales. 
\begin{gather*}
    \uvec = U\uvec',\quad t = \frac{L}{U}t',\quad \x = L\x ',\quad p = \frac{U^2\rho}{L}p',
    \quad b = N^2L\\
    Re = \frac{UL}{\nu}, \quad Pe = \frac{UL}{\kappa}, \quad Ro =
    \frac{U}{2\Omega L}, \quad Fr = \frac{U}{NL}
\end{gather*}

\subsection{Choice of Fast and Slow Spatial/Time Scales}

To create a multiscale model for the statistically steady state of
rotating stratified turbulence, we define the crucial length scales that
specific dynamics occur on. Specifically, we expect both large/slow and
small/fast scales in the horizontal, vertical, and temporal coordinates of the
flow. We take all of our unit coordinates ($x,y,z$) to be defined with respect to the
characteristic horizontal length scale of the flow taken from the stochastic
forcing process ($L_h = 2\pi/\sqrt{2} = \sqrt{2}\pi$). Furthermore, we take our
unit time to be scaled according to the advection timescale of the flow $L_h/U$.
We then define our
small/fast coordinates to be rescaled by the aspect ratio $\alpha = l_z/L_h$,
which compares the small vertical length scale typical of layered anistropic
stratified turbulence to the horizontal forcing length scale $L_h$. 
\begin{gather*}
    \chi = \frac{x}{\alpha}, \quad \xi = \frac{y}{\alpha}, \quad \eta =
    \frac{z}{\alpha}, \quad \tau = \frac{t}{\alpha}
\end{gather*}

Derivative operators are also expanded in order to resolve dynamics over the
various spatio-temporal scales. We have specifically, 
\begin{gather*}
    \grad_{\perp} = \gradhs + \frac{1}{\alpha}\gradhf = \left(\partial_x,
    \partial_y\right)+ \frac{1}{\alpha}\left(\partial_{\chi},
    \partial_{\xi}\right)\\
    \partial_z = \partial_z + \frac{1}{\alpha}\partial_{\eta}\\
    \partial_t = \partial_t + \frac{1}{\alpha}\partial_{\tau}
\end{gather*}

\subsection{Component Expansion of the Primary Variables}
We will continue with a decomposition of our scalar (and each component of our
vector) fields as defined below,
\begin{gather}
    q = \bracket{q}(x,y,z,t) + \bar{q}(x,y,z,\eta,t) +
    q'(x,\chi,y,\xi,z,\eta,t,\tau) \label{eq:flow_components}
\end{gather}
This decomposition introduces a ``bracket'' component which is invariant
under the following isotropic fast-averaging method defined using the same
notation.
\begin{gather}
    \bracket{(\cdot)} \equiv \int_{\tau}\int_{\eta}\int_{\xi}\int_{\chi} (\cdot)
    d\chi d\xi d\eta d\tau \label{eq:fast_avg_iso}
\end{gather}
The ``bar'' component, which varies on the slow scales and only the fast
vertical scale, is invariant under the horizontal fast-averaging method defined
similarly. By contrast, it is not invariant under the ``bracket'' average, as it
averages over the small vertical scales to zero. The ``bracket'' component of the flow
is also invariant under the ``bar'' average as it only varies on the slow
scales. 
\begin{gather}
    \bar{(\cdot)} \equiv \int_{\tau}\int_{\xi}\int_{\chi} (\cdot)
    d\chi d\xi d\tau \label{eq:fast_avg_horiz}
\end{gather}
Finally, the ``prime'' or perturbation component of the decomposition varies on
all of the fast and slow scale variables. This particular component of the
decomposition averages to zero under both of the fast-averaging operators.


\section{Multi-scale expansion of the Governing Equations}


We begin the multi-scale expansion of the governing equations and proceed to
isolate the governing dynamics of each component of our flow by utilizing the
averaging operators. We begin with the expansion of the divergence-free
condition.

\begin{align}
    \grad\cdot\uvec = 0\\
    \left(\grad_{\perp S} + \frac{1}{\alpha}\grad_{\perp
    F}\right)\cdot\left(\bracket{\uvec}_{\perp} + \bar{\uvec}_{\perp} +
    \uvec'_{\perp}\right) + \left(\partial_z +
    \frac{1}{\alpha}\partial_{\eta}\right)\left(\bracket{w} + \bar{w} + w'\right)
    = 0 \\
    \grad_{\perp S} \cdot \bracket{\uvec}_{\perp} + \grad_{\perp S} \cdot
    \bar{\uvec}_{\perp} + \grad_{\perp S} \cdot \uvec'_{\perp} +
    \frac{1}{\alpha}\grad_{\perp F} \cdot \uvec'_{\perp} + \partial_z\bracket{w}
    + \partial_z\bar{w} + \partial_zw' + \frac{1}{\alpha}\partial_{\eta}\bar{w}
    + \frac{1}{\alpha}\partial_{\eta}w'= 0 \label{eq:exp_div}\\
    \bracket{(\ref{eq:exp_div})} \to  \grad_{\perp S} \cdot \bracket{\uvec}_{\perp} +
    \partial_z\bracket{w} = 0 \label{eq:div_1st_avg}\\
    \overline{(\ref{eq:exp_div}) - \bracket{(\ref{eq:exp_div})}} \to \grad_{\perp
    S} \cdot \bar{\uvec}_{\perp} + \partial_z\bar{w} +
    \frac{1}{\alpha}\partial_{\eta}\bar{w} = 0 \\
    (\ref{eq:exp_div}) - \bracket{(\ref{eq:exp_div})} -
    \overline{(\ref{eq:exp_div}) - \bracket{(\ref{eq:exp_div})}} \to \grad_{\perp
    S} \cdot \uvec'_{\perp} + \frac{1}{\alpha}\grad_{\perp F} \cdot
    \uvec'_{\perp} + \partial_zw' + \frac{1}{\alpha}\partial_{\eta}w' = 0 
\end{align}

\subsection{The Buoyancy Equation}

%\begin{equation}
%\begin{split}
\begin{multline}
    \partial_t\bexp +
    \frac{1}{\alpha}\partial_{\tau}b' + \upexp\cdot\left(\grad_{\perp S} +
    \frac{1}{\alpha}\grad_{\perp F}\right)\bexp 
    \\+ \wexp\left(\partial_z +
    \frac{1}{\alpha}\partial_{\eta}\right)\bexp + \wexp \label{eq:b_exp}
    \\
     = \frac{1}{Pe}\left[\left(\grad_{\perp S} + \frac{1}{\alpha}\grad_{\perp F}\right)^2
    \bexp + \left(\partial_z +
    \frac{1}{\alpha}\partial_{\eta}\right)^2\bexp\right]
\end{multline}
%\end{split}
%\end{equation}
%\begin{equation}
%%\begin{split}
\begin{multline}
    \bracket{(\ref{eq:b_exp})} \to \partial_t \bracket{b} +
    \bracket{\uvec_{\perp}}\cdot\gradhs\bracket{b} 
     + \bracket{w}\partial_z\bracket{b} 
     + \bracket{w}
     \\
    = -\gradhs\cdot\bracket{\bar{\uvec}_{\perp} \bar{b}} -
    \gradhs\cdot\bracket{\uvec'_{\perp} b'}-\partial_z\bracket{\bar{w}\bar{b}} - \partial_z\bracket{w'b'}
    + \frac{1}{Pe}\left(\gradhs^2\bracket{b} + \partial_z^2\bracket{b}\right) 
    \label{eq:b_bracket_eq}
\end{multline}

\begin{multline}
    \overline{(\ref{eq:b_exp}) - \bracket{(\ref{eq:b_exp})}} \to \partial_t \bar{b} +
    \left(\bracket{\uvecp} + \bar{\uvec}_{\perp}\right)\cdot\gradhs \bar{b} +
    \bar{\uvecp} \cdot\gradhs\left(\bracket{b} + \bar{b}\right) +
    \left(\bracket{w} + \bar{w}\right)\left(\partial_z +
    \frac{1}{\alpha}\partial_{\eta}\right)\bar{b} 
    \\
    + \bar{w}\partial_z\bracket{b} + \bar{w} 
    = \gradhs\cdot\bracket{\bar{\uvec}_{\perp}\bar{b}}+ 
    \gradhs\cdot\bracket{\uvecp'b'} + \partial_z\bracket{\bar{w}\bar{b}} +
    \partial_z\bracket{w'b'} 
    \\
    - \gradhs\cdot\overline{\uvecp'b'} -
    \partial_z\overline{w'b'} - \frac{1}{\alpha}\partial_{\eta}\overline{w'b'} +
    \frac{1}{Pe}\left(\gradhs^2\bar{b} + \partial_z^2\bar{b} +
    \frac{1}{\alpha^2}\partial_{\eta}^2\bar{b} +
    \frac{2}{\alpha}\partial_z\partial_{\eta}\bar{b}\right)
    \label{eq:b_bar_eq}
\end{multline}

\begin{multline}
    (\ref{eq:b_exp}) - \bracket{(\ref{eq:b_exp})} - \overline{(\ref{eq:b_exp}) -
    \bracket{(\ref{eq:b_exp})}} \to \partial_t b' +
    \frac{1}{\alpha}\partial_{\tau} b' + \left(\bracket{\uvecp} +
    \bar{\uvec}_{\perp} + \uvecp'\right)\cdot\gradhs b' 
    \\
    +\frac{1}{\alpha}\left(\bracket{\uvecp} +
    \bar{\uvec}_{\perp} + \uvecp'\right)\cdot\gradhf b'
    + \uvecp'\cdot\gradhs\left(\bracket{b} + \bar{b}\right)
    + \wexp\partial_zb' 
    \\
    + \frac{1}{\alpha}\wexp\partial_{\eta}b'  +  w'\partial_z\left(\bracket{b} +
    \bar{b}\right) + \frac{1}{\alpha}w'\partial_{\eta}\bar{b}
    +w' = \gradhs\cdot\overline{\uvecp'b'} +
    \partial_z\overline{w'b'} + \frac{1}{\alpha}\partial_{\eta}\overline{w'b'}
    \\
    + \frac{1}{Pe}\left(\gradhs^2b' + \frac{2}{\alpha}\gradhs\gradhf b' +
    \frac{1}{\alpha^2}\gradhf b' + \partial_z^2b' +
    \frac{2}{\alpha}\partial_z\partial_{\eta}b' +
    \frac{1}{\alpha^2}\partial_{\eta}^2b'\right)
    \label{eq:b_pert_eq}
\end{multline}

\subsection{The Vertical Momentum Equation}

\begin{multline}
    \partial_t\wexp +
    \frac{1}{\alpha}\partial_{\tau}w' + \upexp\cdot\left(\grad_{\perp S} +
    \frac{1}{\alpha}\grad_{\perp F}\right)\wexp \\+ \wexp\left(\partial_z +
    \frac{1}{\alpha}\partial_{\eta}\right)\wexp  \label{eq:w_exp}
     = -\left(\partial_z + \frac{1}{\alpha}\partial_{\eta}\right)\pexp +
     \frac{1}{Fr^2}\bexp 
     \\
    + \frac{1}{Re}\left[\left(\grad_{\perp S} + \frac{1}{\alpha}\grad_{\perp F}\right)^2
    \wexp + \left(\partial_z +\frac{1}{\alpha}\partial_{\eta}\right)\wexp\right]
\end{multline}



\begin{multline}
    \bracket{(\ref{eq:w_exp})} \to \partial_t \bracket{w} +
    \bracket{\uvec_{\perp}}\cdot\gradhs\bracket{w} 
     + \bracket{w}\partial_z\bracket{w} = - \partial_z\bracket{p} +
     \frac{1}{Fr^2}\bracket{b}
     \\
    = -\gradhs\cdot\bracket{\bar{\uvec}_{\perp} \bar{w}} -
    \gradhs\cdot\bracket{\uvec'_{\perp} w'}-\partial_z\bracket{\bar{w}\bar{w}} - \partial_z\bracket{w'w'}
    + \frac{1}{Re}\left(\gradhs^2\bracket{w} + \partial_z^2\bracket{w}\right)
    \label{eq:w_bracket_eq}
\end{multline}
%\end{split}
%\end{equation}
%%\begin{equation}
%\begin{split}
\begin{multline}
    \overline{(\ref{eq:w_exp}) - \bracket{(\ref{eq:w_exp})}} \to \partial_t \bar{w} +
    \left(\bracket{\uvecp} + \bar{\uvec}_{\perp}\right)\cdot\gradhs \bar{w} +
    \bar{\uvecp} \cdot\gradhs\left(\bracket{w} + \bar{w}\right)  + 
    \left(\bracket{w} + \bar{w}\right)\left(\partial_z +
    \frac{1}{\alpha}\partial_{\eta}\right)\bar{w}
    \\
    + \bar{w}\partial_z\bracket{w}
    = - \partial_z
    \bar{p} - \frac{1}{\alpha}\partial_{\eta}\bar{p} + \frac{1}{Fr^2}\bar{b}
    + \gradhs\cdot\bracket{\bar{\uvec}_{\perp}\bar{w}}+ 
    \gradhs\cdot\bracket{\uvecp'w'} + \partial_z\bracket{\bar{w}\bar{w}} +
    \partial_z\bracket{w'w'} 
    \\
    - \gradhs\cdot\overline{\uvecp'w'} -
    \partial_z\overline{w'w'} - \frac{1}{\alpha}\partial_{\eta}\overline{w'w'} +
    \frac{1}{Re}\left(\gradhs^2\bar{w} + \partial_z^2\bar{w} +
    \frac{1}{\alpha^2}\partial_{\eta}^2\bar{w} +
    \frac{2}{\alpha}\partial_z\partial_{\eta}\bar{w}\right)
    \label{eq:w_bar_eq}
\end{multline}
\begin{multline}
    (\ref{eq:w_exp}) - \bracket{(\ref{eq:w_exp})} - \overline{(\ref{eq:w_exp}) -
    \bracket{(\ref{eq:w_exp})}} \to \partial_t w' +
    \frac{1}{\alpha}\partial_{\tau} w' + \left(\bracket{\uvecp} +
    \bar{\uvec}_{\perp} + \uvecp'\right)\cdot\gradhs w' 
    \\
    +\frac{1}{\alpha}\left(\bracket{\uvecp} +
    \bar{\uvec}_{\perp} + \uvecp'\right)\cdot\gradhf w'
    + \uvecp'\cdot\gradhs\left(\bracket{w} + \bar{w}\right)
    + \wexp\partial_zw' + \frac{1}{\alpha}\wexp\partial_{\eta}w'  
    \\
    +  w'\partial_z\left(\bracket{w} +
    \bar{w}\right) + \frac{1}{\alpha}w'\partial_{\eta}\bar{w}
     = -\partial_z p' - \frac{1}{\alpha}\partial_{\eta}p' + \frac{1}{Fr^2}b' + 
     \gradhs\cdot\overline{\uvecp'w'} +
    \partial_z\overline{w'w'} + \frac{1}{\alpha}\partial_{\eta}\overline{w'w'}
    \\
    + \frac{1}{Re}\left(\gradhs^2w' + \frac{2}{\alpha}\gradhs\gradhf w' +
    \frac{1}{\alpha^2}\gradhf w' + \partial_z^2w' +
    \frac{2}{\alpha}\partial_z\partial_{\eta}w' +
    \frac{1}{\alpha^2}\partial_{\eta}^2w'\right) \label{eq:w_pert_eq}
\end{multline}

\subsection{The Horizontal Momentum Equation}

\begin{multline}
    \partial_t \upexp + \frac{1}{\alpha}\partial_{\eta} \uvec_{\perp}' +
    \upexp\cdot\left(\gradhs + \frac{1}{\alpha}\gradhf\right)\upexp 
    \\
    + \wexp\left(\partial_z + \frac{1}{\alpha}\partial_{\eta}\right)\upexp +
    \frac{1}{Ro}\left(\ez \times \uexp\right) 
    \\
    = -\left(\gradhs +
    \frac{1}{\alpha}\gradhf\right) \pexp + \F 
    \\
    +\frac{1}{Re}\left[\left(\gradhs +
    \frac{1}{\alpha}\gradhf\right)^2\upexp + \left(\partial_z +
    \frac{1}{\alpha}\partial_{\eta}\right)^2\upexp\right]\label{eq:up_exp}
\end{multline}

\begin{multline}
    \bracket{(\ref{eq:up_exp})} \to \partial_t \bracket{\uvecp} +
    \bracket{\uvec_{\perp}}\cdot\gradhs\bracket{\uvecp} 
     + \bracket{w}\partial_z\bracket{\uvecp} +
     \frac{1}{Ro}\left(\ez\times\bracket{\uvec}\right) = -\gradhs\bracket{p} +
     \bracket{\F}
     \\
     -\gradhs\cdot\bracket{\bar{\uvec}_{\perp} \bar{\uvec}_{\perp}} -
    \gradhs\cdot\bracket{\uvec'_{\perp} \uvec'_{\perp}} -
    \partial_z\bracket{\bar{w}\bar{\uvec}_{\perp}} -
    \partial_z\bracket{w'\uvec'_{\perp}}
    + \frac{1}{Re}\left(\gradhs^2\bracket{\uvecp} + \partial_z^2\bracket{\uvecp}\right)
    \label{eq:up_bracket_eq}
\end{multline}

\begin{multline}
    \overline{(\ref{eq:up_exp}) - \bracket{(\ref{eq:up_exp})}} \to \partial_t
    \baruvecp +
    \left(\bracket{\uvecp} + \bar{\uvec}_{\perp}\right)\cdot\gradhs \baruvecp +
    \baruvecp \cdot\gradhs\left(\brauvecp + \baruvecp\right) 
    + \left(\bracket{w} + \bar{w}\right)\left(\partial_z +
    \frac{1}{\alpha}\partial_{\eta}\right)\baruvecp 
    \\
    + \bar{w}\partial_z\brauvecp = - \gradhs
    \bar{p}  + \bar{\F} 
    + \gradhs\cdot\bracket{\bar{\uvec}_{\perp}\baruvecp}+ 
    \gradhs\cdot\bracket{\uvecp'\perpuvecp} +
    \partial_z\bracket{\bar{w}\baruvecp} +
    \partial_z\bracket{w'\perpuvecp} \\
    - \gradhs\cdot\overline{\uvecp'\perpuvecp} -
    \partial_z\overline{w'\perpuvecp} -
    \frac{1}{\alpha}\partial_{\eta}\overline{w'\perpuvecp} +
    \frac{1}{Re}\left(\gradhs^2\baruvecp + \partial_z^2\baruvecp +
    \frac{1}{\alpha^2}\partial_{\eta}^2\baruvecp +
    \frac{2}{\alpha}\partial_z\partial_{\eta}\baruvecp\right)\label{eq:up_bar_eq}
\end{multline}

\begin{multline}
    (\ref{eq:up_exp}) - \bracket{(\ref{eq:up_exp})} - \overline{(\ref{eq:up_exp}) -
    \bracket{(\ref{eq:up_exp})}} \to \partial_t \perpuvecp +
    \frac{1}{\alpha}\partial_{\tau} \perpuvecp + \left(\bracket{\uvecp} +
    \bar{\uvec}_{\perp} + \uvecp'\right)\cdot\gradhs \perpuvecp +
    \\
    \frac{1}{\alpha}\left(\bracket{\uvecp} +
    \bar{\uvec}_{\perp} + \uvecp'\right)\cdot\gradhf \perpuvecp
    + \uvecp'\cdot\gradhs\left(\brauvecp + \baruvecp\right)
    + \wexp\partial_z\perpuvecp + \frac{1}{\alpha}\wexp\partial_{\eta}\perpuvecp  
    \\
    +  w'\partial_z\left(\brauvecp +
    \baruvecp\right) + \frac{1}{\alpha}w'\partial_{\eta}\baruvecp +
    \frac{1}{Ro}\left(\ez \times \uvec'\right)
     = -\gradhs p' - \frac{1}{\alpha}\gradhf p' + \F' + 
     \gradhs\cdot\overline{\uvecp'\perpuvecp} +
    \partial_z\overline{w'\perpuvecp} 
    \\
    +\frac{1}{\alpha}\partial_{\eta}\overline{w'\perpuvecp}
    + \frac{1}{Re}\left(\gradhs^2\perpuvecp + \frac{2}{\alpha}\gradhs\gradhf
    \perpuvecp +
    \frac{1}{\alpha^2}\gradhf \perpuvecp + \partial_z^2\perpuvecp +
    \frac{2}{\alpha}\partial_z\partial_{\eta}\perpuvecp +
    \frac{1}{\alpha^2}\partial_{\eta}^2\perpuvecp\right) \label{eq:up_pert_eq}
\end{multline}

\subsection{Scaling Arguments for multiscale expansion}


\section{Adding fast time dependence to Bracket Component}

Here we introduce a fast-time dependence to the ``bracket'' component of the
flow variables. 
\begin{gather*}
    \bracket{q} = \bracket{q}(x, y, z, t, \tau)
\end{gather*}
This will affect the leading order of the ``bracket'' equations, i.e. the
fast-time derivative will dominate the dynamics of ``bracket'' components. We
see that equations (\ref{eq:b_bracket_eq}), (\ref{eq:w_bracket_eq}), and
(\ref{eq:up_bracket_eq}) are modified.

\begin{multline}
    (\ref{eq:b_bracket_eq}) \to \partial_t \bracket{b} +
    \frac{1}{\alpha}\partial_{\tau}\bracket{b} + 
    \bracket{\uvec_{\perp}}\cdot\gradhs\bracket{b} 
     + \bracket{w}\partial_z\bracket{b} 
     + \bracket{w}
     \\
    = -\gradhs\cdot\bracket{\bar{\uvec}_{\perp} \bar{b}} -
    \gradhs\cdot\bracket{\uvec'_{\perp} b'}-\partial_z\bracket{\bar{w}\bar{b}} - \partial_z\bracket{w'b'}
    + \frac{1}{Pe}\left(\gradhs^2\bracket{b} + \partial_z^2\bracket{b}\right) 
    \label{eq:new_b_bracket_eq}
\end{multline}

\begin{multline}
    (\ref{eq:w_bracket_eq}) \to \partial_t \bracket{w}
    +\frac{1}{\alpha}\partial_{\tau}\bracket{w} + 
    \bracket{\uvec_{\perp}}\cdot\gradhs\bracket{w} 
     + \bracket{w}\partial_z\bracket{w} = - \partial_z\bracket{p} +
     \frac{1}{Fr^2}\bracket{b}
     \\
    = -\gradhs\cdot\bracket{\bar{\uvec}_{\perp} \bar{w}} -
    \gradhs\cdot\bracket{\uvec'_{\perp} w'}-\partial_z\bracket{\bar{w}\bar{w}} - \partial_z\bracket{w'w'}
    + \frac{1}{Re}\left(\gradhs^2\bracket{w} + \partial_z^2\bracket{w}\right)
    \label{eq:new_w_bracket_eq}
\end{multline}

\begin{multline}
    (\ref{eq:up_bracket_eq}) \to \partial_t \bracket{\uvecp} +
    \frac{1}{\alpha}\partial_{\tau}\bracket{\uvecp} + 
    \bracket{\uvec_{\perp}}\cdot\gradhs\bracket{\uvecp} 
     + \bracket{w}\partial_z\bracket{\uvecp} +
     \frac{1}{Ro}\left(\ez\times\bracket{\uvec}\right) = -\gradhs\bracket{p} +
     \bracket{\F}
     \\
     -\gradhs\cdot\bracket{\bar{\uvec}_{\perp} \bar{\uvec}_{\perp}} -
    \gradhs\cdot\bracket{\uvec'_{\perp} \uvec'_{\perp}} -
    \partial_z\bracket{\bar{w}\bar{\uvec}_{\perp}} -
    \partial_z\bracket{w'\uvec'_{\perp}}
    + \frac{1}{Re}\left(\gradhs^2\bracket{\uvecp} + \partial_z^2\bracket{\uvecp}\right)
    \label{eq:new_up_bracket_eq}
\end{multline}


\end{document}

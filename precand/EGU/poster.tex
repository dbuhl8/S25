\documentclass[30pt, blockverticalspace=1cm]{tikzposter}
\geometry{paperwidth=1978mm,paperheight=1183mm}
\makeatletter
\setlength{\TP@visibletextwidth}{\textwidth-2\TP@innermargin}
\setlength{\TP@visibletextheight}{\textheight-2\TP@innermargin}
\makeatother
%\documentclass[]{beamer}



\title{\parbox{\linewidth}{\centering Rotating Stratified Turbulence
}}
\author{Dante Buhl, Pascale Garaud, Hongyun Wang}
\date{April 29, 2025}
\institute{University of California Santa Cruz}
\usetheme{Default}
\usecolorstyle{Default}
\usecolorpalette{Default}

%% PACKAGES %%
\usepackage[utf8]{inputenc}
\usepackage{amsmath, amssymb, amsthm}
\usepackage{mathtools}
\usepackage[shortlabels]{enumitem}
\usepackage{tikz}
\usepackage{subfig}
\usepackage{verbatim}
\usepackage{enumerate}
\usepackage{graphicx}
\usepackage{bm}

%for positioning nodes
\usetikzlibrary{positioning}
\usepackage[none]{hyphenat}


\begin{document}


\newcommand{\red}{\color{red}}
\newcommand{\wrms}{w_{\text{rms}}}
\newcommand{\bs}[1]{\boldsymbol{#1}}
\newcommand{\tb}[1]{\textbf{#1}}
\newcommand{\bmp}[1]{\begin{minipage}{#1\linewidth}}
\newcommand{\emp}{\end{minipage}}
\newcommand{\R}{\mathbb{R}}
\newcommand{\C}{\mathbb{C}}
\newcommand{\N}{\mathcal{N}}
\newcommand{\K}{\bs{\mathrm{K}}}
\newcommand{\m}{\bs{\mu}_*}
\newcommand{\s}{\bs{\Sigma}_*}
\newcommand{\dt}{\Delta t}
\newcommand{\dx}{\Delta x}
\newcommand{\tr}[1]{\text{Tr}(#1)}
\newcommand{\Tr}[1]{\text{Tr}(#1)}
\newcommand{\Div}{\nabla \cdot}
\renewcommand{\div}{\nabla \cdot}
\newcommand{\Curl}{\nabla \times}
\newcommand{\grad}{\nabla}
\newcommand{\pt}{\partial t}
\newcommand{\pz}{\partial z}
\newcommand{\uvec}{\bs{u}}
\newcommand{\F}{\bs{F}}
\newcommand{\T}{\tilde{T}}
\newcommand{\ez}{\bs{e}_z}
\newcommand{\ex}{\bs{e}_x}
\newcommand{\ey}{\bs{e}_y}
\newcommand{\eo}{\bs{e}_{\bs{\Omega}}}
\newcommand{\ppt}[1]{\frac{\partial #1}{\partial t}}
\newcommand{\DDt}[1]{\frac{D #1}{D t}}
\newcommand{\dd}[2]{\frac{d #1}{d #2}}
\newcommand{\pp}[2]{\frac{\partial #1}{\partial #2}}
\newcommand{\ddz}[1]{\frac{d #1}{d z}}
\newcommand{\ppx}[1]{\frac{\partial #1}{\partial x}}
\newcommand{\ppy}[1]{\frac{\partial #1}{\partial y}}


%% images in corners %%
% need to add UCSC logo
\node [below right=-1cm and 5cm] at (bottomleft |- topright)
{\includegraphics[width=21cm]{ucsc_logo_2.png}};
\node [below left=-.5cm and 7cm] at (topright) {\includegraphics[width=15cm]{images/NSF-logo.png}};

%% TITLE AND LOGOS %%
\maketitle

\begin{columns}


%%%%%%%%%%%%%%%%%%%%%%%%%%%%%%%%%%%%%%%%%%%%%%%%%%%%%%%%
\column{0.33}
%%%%%%%%%%%%%%%%%%%%%%%%%%%%%%%%%%%%%%%%%%%%%%%%%%%%%%%%


\block{Abstract}
{
Recent interest in the dynamics of stratified turbulence has led to the
development of new models for quantifying vertical transport of momentum and
buoyancy (Chini {\em et al} 2022, Shah {\em et al} 2024). These models are still
incomplete as they do not yet include all of the relevant dynamics often present
in real physical settings such as rotation and magnetic fields. Here we expand
on prior work by adding rotation. We conduct 3D direct numerical simulations of
rotating, stochastically forced, strongly stratified turbulence ($Fr \ll  1$), and vary the Rossby number. We find that rotation gradually suppresses small-scale 3D motions and therefore inhibits vertical transport as Ro decreases towards Fr. The effect is particularly pronounced within the cores of emergent cyclonic vortices. For sufficiently strong rotation, vertical motions are entirely suppressed.
}


\block{Motivation}
{ 
    % TAKE FROM DFD PRESENTATION
\bmp{.47}
    
    Nonrotating stratified turbulence is characterized by strongly
    anisotropic pancake structures within the flow. 

    \includegraphics[width=.9\linewidth]{images/pancakes.pdf}
     %end small
\emp
\hfill
\bmp{.47}
    
    Rotation promotes barotropic structures which are invariant along the axis
    of rotation.

    \includegraphics[width=.9\linewidth]{images/cylinders.pdf}
     %end small
\emp
}


\block{The Equations}
{
\centering
\bmp{.4}
\begin{gather*}
    \DDt{\uvec} + \frac{1}{Ro}\ez\times\uvec = -\grad p + \F + \frac{1}{Fr^2}T \ez 
    + \frac{1}{Re} \grad^2 \uvec
\end{gather*}
\begin{gather*}
    \DDt{T} + w = \frac{1}{Pe} \grad^2T, \quad \div{\uvec} = 0
\end{gather*}
\begin{gather*}
    Re = \frac{UL}{\nu},\quad  Pe = \frac{UL}{\kappa_T},\quad Fr =
    \frac{U}{NL},\quad  Ro =
    \frac{U}{2\Omega L}
\end{gather*}
\emp
\bmp{.6}
\centering
\includegraphics[width=.9\linewidth]{images/schematic.pdf}
\emp
}
\block{Stochastic Forcing}{

    \bmp{.45}
        We choose the forcing to be purely horizontal and divergence-free stochastic
        process:
        \[
            \bs{F} = F_x\ex + F_y\ey, \quad \grad \cdot \bs{F} = 0
        \]
        The forcing is applied in spectral space and satisfies $\bs{k} \cdot \hat{\bs{F}} = 0$:
        \begin{gather*}
            \hat{F}_x = \frac{k_y}{|\bs{k}_h|}G(\bs{k}_h,t), \quad \hat{F}_y = \frac{-k_x}{|\bs{k}_h|}G(\bs{k}_h,t)\\
        \end{gather*}
        where $G(\bs{k}_h,t)$ is a Gaussian process of amplitude 1 and correlation
        timescale 1, and $|\bs{k}_h| \le \sqrt{2}$. The gaussian process
        timeseries are
        autoregressively generated in spectral space using a pseudo-inverse
        algorithm. The gaussian kernel is generated using the exponential
        squared kernel function
        \begin{gather*}
            K(t, t') = \exp\left(\frac{-(t - t')^2}{2\tau^2}\right)
        \end{gather*}
        in order to improve smoothness and differentiability of the forcing in
        time. The image on the right depicts an example of the forcing in time
        for a single wavenumber taken from one of the non-rotating simulations. 

        \begin{flushright}
        c.f. Waite and Bartello (2004)\end{flushright}
    \emp
    \bmp{.55}
        \centering
        \includegraphics[width=.95\linewidth]{images/forcing_example.pdf}
    \emp
}

%%%%%%%%%%%%%%%%%%%%%%%%%%%%%%%%%%%%%%%%%%%%%%%%%%%%%%%%
\column{0.33}
%%%%%%%%%%%%%%%%%%%%%%%%%%%%%%%%%%%%%%%%%%%%%%%%%%%%%%%%


\block{Simulation Visualization}{
    % start showing data and results. Things to include in this block should be
    % quantitative datas, rms scaling laws with rossby and the such

    Two series of simulations have been conducted each with a different fixed
    Froude number. The following volume renderings and spectra plots are taken
    from simulations with $Fr \approx 0.18$. These volume renderings demonstrate
    the affect of rotation on the general structure of the flow field and where
    thermal mixing occurs in the flow. The top row depicts the vertical
    component of vorticity for simulations with different input Rossby numbers
    (decreasing from left to right). The vorticity renderings depict the
    formation of cyclones and anticyclones which exhibit increasing vertical
    invariance as the global rotation becomes more rapid. Correlated with the
    appearance of these cyclones and anticyclones, is the apparent confinement
    of thermal dissipation within the anticyclones of the flow field (second
    row). Alongside this reorganization of thermal transport and mixing, we also
    note the appearance of an inverse energy cascade to the largest scales of
    the flow. This is demonstrated by examining the horizonatal (red) and
    vertical (blue) energies of the flow at various timesteps according to the
    horizontal wavenumber (bottom row). It should be noted that since the
    selected forcing injects energy up to $|\bs{k}_h| \le \sqrt{2}$, there is a
    plateau in the horizontal energy at low wavenumbers. Plotted over the
    spectra are a $\bs{k}_h^{-5/3}$ decay scaling (solid red line) and $|\bs{k}_h| =
    \sqrt{2}$ marking the smallest scale that energy is forced onto (dashed blue
    line). 


    \begin{center}

    % this may be the appropriate section to include the QR code. 
    \bmp{.09}
        
        \centering
        {\Large $\omega_z$}
    \emp
    \bmp{.3}
        \centering
        {\Large $Ro = 2$}
        \vspace{10pt}

        \includegraphics[width=.85\linewidth]{images/vortz_Om0.5_vr2.png}
    \emp
    \bmp{.3}
        \centering
        {\Large $Ro = .5$}

        \vspace{10pt}
        \includegraphics[width=.85\linewidth]{images/vortz_Om2_vr2.png}
    \emp
    \bmp{.3}
        \centering
        {\Large $Ro = .1$}

        \vspace{10pt}
        \includegraphics[width=.85\linewidth]{images/vortz_Om10_vr2.png}
    \emp

    \bmp{.09}
        \centering
        {\Large $|\nabla T|^2$}
    \emp
    \bmp{.30}
        \centering
        \includegraphics[width=.85\linewidth]{images/chi_Om0.5_vr2.png}
    \emp
    \bmp{.30}
        \centering
        \includegraphics[width=.85\linewidth]{images/chi_Om2_vr2.png}
    \emp
    \bmp{.30}
        \centering
        \includegraphics[width=.85\linewidth]{images/chi_Om10_vr2.png}
    \emp

    \bmp{0.09}
        \centering
        {\Large Energy Spectra}
    \emp
    \bmp{.3}
        \centering
        \includegraphics[width=.9\linewidth]{images/Om0.5Spec.pdf}
    \emp
    \bmp{.3}
        \centering
        \includegraphics[width=.9\linewidth]{images/Om2Spec.pdf}
    \emp
    \bmp{.3}
        \centering
        \includegraphics[width=.9\linewidth]{images/Om5Spec.pdf}
    \emp
    \end{center}
}

\block{RMS Data}{
    \centering

    \bmp{.3}
    From simulations which have reached a statistically stationary state
    (filled circles) we can observe adjustments to the rms horizontal and
    vertical velocities (top left and right) and to the rms temperature
    flux and mixing efficiency (bottom left and right) as the inverse Rossby number
    varies. Each of these quantities react differently to the
    changes by the Rossby number, however, most prominent is the increase in the
    average horizontal velocity and appears to scale by twice the inverse Rossby
    number. The vertical velocity in the nonrotating case should scale by the
    Froude number ({\red CITE CHINI ET AL HERE}). This scale separation between
    simulation sets becomes less distinct as the inverse Rossby number
    increases. Adjustments to the temperature flux and mixing efficiency are
    less much less prominent, and the effect of rotation on these quantities is
    unclear from these plots.

    \emp
    \bmp{.7}
        \centering
        \includegraphics[width=\linewidth]{images/subplots.pdf}
    \emp
}


%%%%%%%%%%%%%%%%%%%%%%%%%%%%%%%%%%%%%%%%%%%%%%%%%%%%%%%%
\column{0.33}
%%%%%%%%%%%%%%%%%%%%%%%%%%%%%%%%%%%%%%%%%%%%%%%%%%%%%%%%

\block{Vertically Averaged Quantities}{

    We use
    vertical averaging to visualize and compare the local rotation of the flow
    to average temperature flux, thermal dissipation, and rms vertical velocity
    within the corresponding pencils.
    A likely quantity to
    correlate empirical flow data with regions of thermal mixing and vertical
    transport appears to be the local rotation rate of the flow, 
    \begin{gather*}
        Ro_{\text{eff}}^{-1} = |\omega_z + Ro^{-1}|
    \end{gather*}
    This method of correlation is heavily dependent on the colorbar used to
    measure the local rotation rate. Without a sufficient theoretical model to
    test, so far, we only have empirical results which appear to satisfy the
    data to some level of accuracy. 

    \begin{center}
    \bmp{.33}
        \centering
        \includegraphics[width=\linewidth]{images/Om0.5B100Re600Pe60_vert_avg.pdf}
    \emp
    \bmp{.33}
        \centering
        \includegraphics[width=\linewidth]{images/Om2B100Re600Pe60_vert_avg.pdf}
    \emp
    \bmp{.33}
        \centering
        \includegraphics[width=\linewidth]{images/Om8B100Re600Pe60_vert_avg.pdf}
    \emp


    \end{center}

    \begin{gather*}

    \end{gather*}
}


\block{References \& Acknowledgements}{
    This work is supported by NSF {\red grant number here}. 

    \
}

\end{columns}


\end{document}

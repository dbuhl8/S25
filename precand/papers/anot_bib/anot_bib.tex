\documentclass{article}

\usepackage{graphicx} % Required for inserting images
\usepackage[left=1in,right=1in,top=1in,bottom=1in]{geometry}
\usepackage{amsmath}
\usepackage{amsthm} %proof environment
\usepackage{amssymb}
\usepackage{amsfonts}
\usepackage{enumitem} %nice lists
\usepackage{verbatim} %useful for something 
\usepackage{xcolor}
\usepackage{setspace}
\usepackage{blindtext} % I have no idea what this is 
\usepackage{caption}  % need this for unnumbered captions/figures
\usepackage{natbib}
\usepackage{tikz}
\usepackage{hyperref}

\begin{document}

\title{Pre-candidacy notes: }
\author{Dante Buhl}

\newcommand{\wrms}{w_{\text{rms}}}
\newcommand{\bs}[1]{\boldsymbol{#1}}
\newcommand{\tb}[1]{\textbf{#1}}
\newcommand{\bmp}[1]{\begin{minipage}{#1\textwidth}}
\newcommand{\emp}{\end{minipage}}
\newcommand{\R}{\mathbb{R}}
\newcommand{\C}{\mathbb{C}}
\newcommand{\N}{\mathcal{N}}
\newcommand{\K}{\bs{\mathrm{K}}}
\newcommand{\m}{\bs{\mu}_*}
\newcommand{\s}{\bs{\Sigma}_*}
\newcommand{\dt}{\Delta t}
\newcommand{\dx}{\Delta x}
\newcommand{\tr}[1]{\text{Tr}(#1)}
\newcommand{\Tr}[1]{\text{Tr}(#1)}
\newcommand{\Div}{\nabla \cdot}
\renewcommand{\div}{\nabla \cdot}
\newcommand{\Curl}{\nabla \times}
\newcommand{\Grad}{\nabla}
\newcommand{\grad}{\nabla}
\newcommand{\grads}{\nabla_s}
\newcommand{\gradf}{\nabla_f}
\newcommand{\xs}{x_s}
\newcommand{\xf}{x_f}
\newcommand{\ts}{t_s}
\newcommand{\tf}{t_f}
\newcommand{\pt}{\partial t}
\newcommand{\pz}{\partial z}
\newcommand{\uvec}{\bs{u}}
\newcommand{\F}{\bs{F}}
\newcommand{\T}{\tilde{T}}
\newcommand{\ez}{\bs{e}_z}
\newcommand{\ex}{\bs{e}_x}
\newcommand{\ey}{\bs{e}_y}
\newcommand{\eo}{\bs{e}_{\bs{\Omega}}}
\newcommand{\ppt}[1]{\frac{\partial #1}{\partial t}}
\newcommand{\DDt}[1]{\frac{D #1}{D t}}
\newcommand{\ppts}[1]{\frac{\partial #1}{\partial t_s}}
\newcommand{\pptf}[1]{\frac{\partial #1}{\partial t_f}}
\newcommand{\ppz}[1]{\frac{\partial #1}{\partial z}}
\newcommand{\ddz}[1]{\frac{d #1}{d z}}
\newcommand{\ppzetas}[1]{\frac{\partial^2 #1}{\partial \zeta^2}}
\newcommand{\ppzs}[1]{\frac{\partial #1}{\partial z_s}}
\newcommand{\ppzf}[1]{\frac{\partial #1}{\partial z_f}}
\newcommand{\ppx}[1]{\frac{\partial #1}{\partial x}}
\newcommand{\ppxi}[1]{\frac{\partial #1}{\partial x_i}}
\newcommand{\ppxj}[1]{\frac{\partial #1}{\partial x_j}}
\newcommand{\ppy}[1]{\frac{\partial #1}{\partial y}}
\newcommand{\ppzeta}[1]{\frac{\partial #1}{\partial \zeta}}


\maketitle 
\tableofcontents
\pagebreak
% This line removes the automatic indentation on new paragraphs
\setlength{\parindent}{0pt}

\section{Billant \& Chomaz Papers}
    \subsection{Experimental evidence for a new instability of a
    vertical columnar vortex pair in a strongly
    stratified fluid (2000)}
    \begin{itemize}
        \item The first paper in a series of papers by Billant and Chomaz describing
        and investigating the properties of the so called ``zigzag'' instability
        present in the Lamb-Chaplygin vortex pair (a counterrotating vortex dipole). 
        \item This paper demonstrated the existence of such an instability from
        experimental findings at sufficient stratification. For insufficient
        stratification $Fr \ge 0.2 \pm 0.01$, the ellipitcal instability appears to
        be the dominant instability and after its gravitational collapse, the vortex
        pair appears to irregularly zigzag into layer formation. 
        \item From what can be observed from the zigzag instability is that it
        doesn't perturb the horizontal cross-section structure of the vortex column,
        only its vertical structure. It is positted that this phenomenon may be
        responsible for the layering phenomenon demonstrated in many stratified
        flows. 
        \item Over a long enough time frame the original vortex pair column ends up
        divided into pancake dipole segments in the vertical direction, obtaining
        what is usually described as pancake eddies in the flow. 
    \end{itemize}

    \subsection{Self-similarity of strongly stratified inviscid flows (2001)}

    \begin{itemize}
        \item Posits the scaling of an intrinsic vertical length scale of strongly
        stratified flows, $l_z \propto U/N$. 
        \item Third paper which describes the ``Zig-Zag'' instability. Two previous
        papers conducted linear stability analysis of the instability. 
        \item Zig-zag instability is self-similar with respect to $k_zU/N$ which
        implies that the dominant vertical wavenumber of the flow is proportional to
        $Fr$. 
    \end{itemize}

    \subsection{Three-dimensional stability of vertical columnar vortex pair in a
    stratified fluid}

    \begin{itemize}
        \item This paper conducted a numerical stability analysis on the linearized
        equations using mean-pertubation flow separation. They found for flows
        with sufficient stratification that the primary instability of the
        counterrotating vortex pair was the ``zig-zag'' instability in which the
        entire vortex column was destabilized and oscillationed ide to side with a
        typical scale height, later found to be proportional to the froude number.
        For insufficient stratification, the elliptical instability was the dominant
        instabtility. 
        \item Among their findings is the approximate scaling that the root mean
        squared $u_z' \propto
        1/Fr$ and $p' \propto Fr$ (normalized by the rms horizontal velocity).
        Furthermore, these numerical findings for the growth rate of the zig-zag
        instability concur with the experimental results
        within reasonable error. 
        \item Their nondimensionalization involved 
        \begin{gather*}
            Fr = \frac{U_{\text{prop}}}{NR}
        \end{gather*}
        where $U_{\text{prop}}$ is the propagation speed of the vortex pair, and $R$
        is given by the dipole radius. This is similar to the non-dimensionalization
        from Chini et al, in which the unit velocity and lengthscale are given by
        the typical horizontal flow (i.e. horizontal forcing which is order 1 in
        both $U$ and $L$). 
    \end{itemize}

    \subsection{Three-dimensional stability of a horizontally
sheared flow in a stably stratified fluid (Deloncle et al 2007)}
    \begin{itemize}
        \item Inviscid non-rotating flow with a horizontal shear background
        flow, and a vertical stable stratification will have a primary
        instability which is two-dimensional, but as the Froude number tends
        toward zero, they find that an increasing number of three-dimensional
        instability modes are destabilized and may invalidate Squire's theorem
        (1933) for this given background flow.
    \end{itemize}

    \subsection{Three-dimensional stability of vortex arrays
in a stratified and rotating fluid (Deloncle et al (2011)}
    \begin{itemize}
        \item They find that the wavenumber of the primary instability scales
        with $Ro/(bFr)$, where $b$ is the separation distance between the vortex
        columns. 
        \item Rossby and Froude numbers are defined according to the circulation
        of the vortices imposed at the IC. 
        \item Asymptotic analysis of perturbations to the center of vortex
        cores. This portion I don't fully understand and am not quite sure what
        they are getting at here. It seems to be calling back to the fact that
        the zigzag instability perturbs the vertically coherent cyclones. 
        \item Numerical Simulations are performed using a pseudo-spectral code.
        There are two sections of simulations performed. First an initialization
        of the base state, whereby and initial vorticity distribution is evolved
        until it reaches a quasi-steady state. Then a code linearized around
        that state is ran and looks at the perturbations to the flow. After a
        significant amount of simulation time, the dominant mode emerges and
        they can look into the structure and properties of that mode. 
        \item 
    \end{itemize}


\section{Hattori \& Hirota Papers}
    \subsection{Stability of two-dimensional Taylor-Green vortices in rotating
    stratified fluids (2023)}
    \begin{itemize}
        \item Conducted a local stability analysis as well as DNS and analyzed the
        data using modal stability analysis. 
        \item Linear Stability analysis is conducted on a linearized and inviscid
        version of the governing equations.
        \item Both the DNS and LSA begin with a base flow composed of Taylor-Green
        vortices, which are arranged in a grid lattice. 
        \item 5 instabilities are identified from the LSA, each with a different
        mechanism and different instability/resonance conditions. 
        \item Linear Stability analysis found that the pure hyperbolic instability
        is often the fastest growing istability as also the most realizable.
        Variation of the input rossby and froude numbers reveals characteristics of
        other secondary instabilities which vary with vertical wavenumber, and
        radius from vortex centers (as well as input parameters). 
    \end{itemize}

    \subsection{Modal stability analysis of arrays of stably stratified vortices
    (2021)}
    \begin{itemize}
        \item This paper conducted a modal stability analysis for nonrotating
        stratified Taylor-Green (and Stuart) vortices. They find three vital
        hyperbolic instabilities which are primarily affected by stratification:
        the pure hypebolic, strato-hyperbolic, and mixed-hyperbolic instability. 
        \item One of the distinguishing metrics for these instabilities are
        two integral quantities $s_1$, $s_2$. 
        \begin{gather*}
            s_1 = \frac{\int_{|\Phi| \le 0.1b} \omega_z'^2dV}{\int_V \omega_z^2
            dV}\\
            s_2 = \frac{\int_V \left(\div_h\uvec_h'\right)^2dV}{\int_V \omega_z^2
            dV}\\
        \end{gather*}
        The stratohyperbolic instability is characterized by zero or near zero
        values of $s_1$ ($< 0.1$). The pure hyperbolic instability is characterized by low
        values of $s_2$ ($< 0.1$) and intermediate values of $s_1$ ($\sim 0.3$).
        Finally, the mixed hyperbolic instability is seemingly unbounded by
        either $s_1$ or $s_2$ (c.f. Figure 7 for visual representation). 
        \item The modal stability analysis shows that the fastest growing mode
        and instability of the taylor green vortices in the stratified case is
        always the mixed hyperbolic instability and the fastest growing mode is
        consistently near $k_zL_0 \approx 10$ (multiplied by $L_0$ to
        non-dimensionalize $k_z$). 
        \item This paper also discusses and demonstrates the modal stuctures of
        these instabilities. Most important of these characteristics is its
        vertical structure, however, there are modes with nontrivial structures
        such as the mixed-hyperbolic ``spiral-mode'' which very closely
        resembles the structure of the vertical velocity from my own rotating
        stratified simulations (within the stable cyclones in the flow). 
    \end{itemize}

\subsection{Suzuki et al (2018), Strato-hyperbolic instability: a new mechanism
of instability in stably stratified vortices}
    \begin{itemize}
        \item Need to read this, makes comparisons between the stratohypberolic
        and zigzag instability
    \end{itemize}


\section{Miyazaki and Fukumoto}
    \subsection{Three‐dimensional instability of strained vortices in a stably
    stratified fluid (1992)}
    \begin{itemize}
        \item This paper conducts a linear stability analysis of ``unbounded
        strained vortices''. The linear stability analysis is derived analytically
        and then solved numerically by a floquet problem. The primary investigation
        is into the elliptical instability of ``Pierrehumbertt type''. Two other
        instability modes are noted which depend distinctly on the buoyancy
        frequency $N$.
    \end{itemize}

    \subsection{Elliptical instability in a stably stratified rotating fluid
    (1993)}
    \begin{itemize}
       \item This paper studies the elliptical instability present for
       vorticies with an aspect ratio between the major and minor
       axes of the ellipse created by the streamlines of the flow 
       $\varepsilon \ne 1$. 
       \item The problem assumed the flow is inviscid, incompressible, and
       nondiffusive. Their nondimensionalization relies on the following:
       \begin{gather*}
            Ro = \frac{U}{\Omega L} = \frac{\gamma}{\Omega}, \quad Fr =
            \frac{U}{NL} = \frac{\gamma^2 L}{g}
       \end{gather*}
       {\color{red} the paper may have a typo in the definition for $Ro$}
       \item The problem is solved numerically using a linearized
       mean-perturbation equation. The perturbation equations assume an ansantz
       of fourier modes with a floquet multiplier and then are numerically
       intergrated to determine the growth rate for varying rotation and
       stratification. 
       \item The paper divides this study into sections where first the effects
       of stratification are ignored (similar to the 1992 papers), and then
       rotation is ignored. Finally, they cover some cases where both rotation
       and stratification are included and characterize phenomenon of the
       elliptical instability. 
       \item In the purely rotating case, specifically for strong cyclones, we see that there is a subharmonic and
       superharmonic mode of the elliptical instability where the superharmonic
       mode is very weak and exists for a very small range of vertical
       wavenumbers. For strong anti-cyclones, we find that the superharmonic mode
       disappears and the subharmonic mode becomes very realizable as
       horizontal disturbances destabilize the vortex. 
       \item When the anticyclones are weaker ($|Ro|$ is larger), the elliptical
       instability's growth rate resembles that for the cyclones. 
       \item Weak cyclones/anticyclones in the presence of stratification
       and roation are very stable to the elliptical instabtility. Roughly we
       can find if a particular strained vortex is unstable by seeing if its
       initial $\varepsilon_0^*$ has any unstable modes. If not, it may be the
       case that that particular vortex is stable in a specific configuration of
       $Fr$ and $Ro$. 
       \item {\color{red} mention their conclusion}
    \end{itemize}

\section{Linden Papers}
    \subsection{The stability of vortices in a rotating, stratified fluid}
        \begin{itemize}
            \item experimental setup where two layer vortices are created in a
            closed volume. A rotating tank whose center is aligned with the
            axis of rotation, is spun up to a steady state and then a cylinder
            immersed at a depth $H$ in the center is removed which contains dye.
            When spun up, this 
            cylinder will be rotating, and form a vortex in the flow (either at
            the surface or at the bottom according to the density regime
            ($\rho_1 > \rho_2$ or $\rho_1 < \rho_2$)
            \item These closed volume vortices lose their intial structure over
            time, whereby the dyed vortex will disperse radially in either an
            axissymmetric or non-axissymmetric manner (need to clarify what
            causes this)
            \item Essentially, the disparity between the density layers causes
            the vortex to become unstable and move (either upwards to downwards
            and then radially)
            according to the stratification. This movement is opposed by the
            conservation of angular momentum, and we see either cyclonic or
            anticyclonic movements within the vortex according to whether the
            vortex expands or contracts radially. 
            \item Non-axissymmtric movements are then induced due to a
            baroclinic instability which fueled by potential energy stored
            in the density gradient and by the kinetic energy contained in
            the horizontal shear in the flow. 
            \item The number of vortices shed radially from the original vortex
            is given by a modal configuration $n$, which increases with the
            depth ratio of the vortex. That is, for smaller vortex depth ratios,
            the fastest growing unstable mode is $n=2$, whereas for larger depth
            ratios the fastest growing unstable mode is $n=3$ (three
            shed vortices). 
            \item The shed vortices are in a dipole configuration, similar to
            lamb-Chaplygin vortex pairs.
        \end{itemize}

\section{Herring and Metias}
    \subsection{}
    \begin{itemize}
        \item stuff
    \end{itemize}

\section{Waite and Bartello}
    \subsection{}
    \begin{itemize}
        \item stuff
    \end{itemize}


\section{GFD Group (Garaud, Chini, Shah, Caulfield \ldots)}
    \subsection{Exploiting self-organized criticality in strongly stratified
    turbulence (2021)}
    \begin{itemize}
        \item Developed a multiscale model for strongly stratified flows wherein an
        aspect ratio $\alpha$ is used to describe scale separation of horizontal and
        vertical motions recovering that $l_z \propto Fr$ as posited by
        %\cite{BillantChomaz2001}. 
    \end{itemize}

    \subsection{Cope et al. 2020}

    \subsection{Shah et al. 2023}

    \subsection{Layering, Instabilities, and Mixing in Turbulent Stratified
    Flows, Caulfield 2021}
    \begin{itemize}
        \item This is a review paper aimed at highlighting revcent findings in
        density stratified flows as well as the importance of relevant
        length scales, energy conversion mechanisms, and thermal/compoisitional
        mixing in the flow. 
        \item This paper dicusses many aspects of stratified flow. Some of the
        most fundamental quantities which are still trying to be understood are
        the mixing efficiency and turbulent flux coefficient as they depend on
        flow parameters. The complication with this conceptualization, is that
        it is unclear which flow paremeters are the right ones, and more
        importnatly if the flow parameters exhibit dependence on one another. 
        \item Caulfield also details several relevant lengthscales intrinsic to
        stratified turbulence. Among these scales are a typical horizonal
        length scale of the background flow $l_h$, the vertical length scale
        from the anisotropic flow field $l_v$, the Ozmidov Length scale $l_O$
        which describes the length scale seperating anisotropic dynamics from
        the isotropic dynamics, and finally the Kolmorogov length scale $l_K$
        which describes the smallest length scale where isotropic turbulence
        occurs. According to Caulfield, in the ``Layered Anisotropic Stratified
        Turbulence'' (LAST) regime there is an implicit asymptotic ordering of
        these length scales given by, 
        \begin{gather*}
            l_K \ll l_O \ll l_v \ll l_h
        \end{gather*}   
    \end{itemize}

\section{Praud, Sommeria, Fincham}
    \begin{itemize}
        \item Perform experiments of rotating grid turbulence in a strongly
        stratified fluid. Using a scanning-image velocimetry technique, they
        measure the flow field at discrete fime points. 
        \item Generally, they find as quasi-geostrophy
        predicts that a direct energy cascade is inhibitted. In the limit of
        small froude and Rossby numbers, they find that the the flow is
        non-dissipative and associated with an intermitant vorticity vield and
        exhibits a $k_h^{-3}$ energy spectrum. 
        \item For higher Rossby numbers, there are differences from the
        Quasi-geostropic mnodel. A decay of kinetic energy (not dependent on Fr)
        is observed. Both cyclones and anti-cyclones occur in the flow, but only
        cyclones are specifically intense (high-amplitude). At later times, the
        flow is composed of ``lens-like'' eddies which have an aspect ratio
        $f/N$. 
        \item They find similar to our DNS, that the flow is dominated by
        cyclonic vortices and for strongly rotating flows also anticyclones. For
        the moderately rotating case, the regions outside of the cyclones cannot
        find geostropic balance and are not able to form coherent structure. 
        \item They definte their Fr Ro and Re, 
        \begin{gather*}
            Fr_M = \frac{U}{MN}, \quad Ro_M = \frac{U}{Mf}, \quad Re_M =
            \frac{UM}{\nu}
        \end{gather*}
    \end{itemize}

\section{Guo, Taylor, Zhou}
    \subsection{Zigzag instability of columnar Taylor-Green vortices in a
    strongly stratified fluid}
    \begin{itemize}
        \item This paper performs a linear stability analysis of the primary
        instability of the Taylor-Green vortex columns. They find that this
        instability is the zig-zag instability, whereby it develops with a
        vertical lengthscale and wavenumber which scale with $Fr^{-1}$ as shown
        by Billant and Chomaz in their 2001 papers on the same instability. 
        \item They also detail in a similar manner to the Hattori papers, the
        modal structure of this fastest growing unable mode. Note that this
        confirms that the mixed hyperbolic instability presented by the Hattori
        papers (modal stability analysis of 2d taylor green vortices 2023) is in
        fact the zigzag instability, which is reliably the fastest growing
        instability for the 2D Taylor-Green vortices in a stratified fluid. 
        \item The findings of this paper support the claim that the zig-zag
        instability is a fundamental instability which transfers energy from
        vertically invariant modes to three-dimensional turbulence characterized
        by the Froude number. 
        \item Notes on the study, this linear stability analysis shows only the
        development of instability from the vortex columns and does not include
        the effect of rotation in their study. 
        \item They compute the most dominant lengthscale of the flow according
        to an integrl over wavenumbers of the kinetic energy fo the flow, i.e.
        grouping of energy is what they use to declare the dominant vertical
        lengthscale of the instabiltiy (c.f. eqn 3.4)
        \item 
    \end{itemize}

\section{Potylitsin and Peltier}
    \subsection{Stratification effects on the stability of columnar
    vortices on the f-plane}
    \begin{itemize}
        \item Rapidly rotating, unstratified vortices are unstable to the
        centrifugal instability. Stratification stabilizes this instability and
        other modes in the case of no rotation. 
        \item For anti-cyclones destructive instabilities arise for vortices of
        high ellipticity. 
        \item Quote from introduction on page 2 of the PDF, ``that two-dimensional vortices do
not sense the background rotation as long as the influence of rotation is considered
from an f-plane perspective and the axis of rotation is aligned with the axis of the
vortices.''
        \item Quote form page three of the pdf, ``The conditions in which three-dimensional centrifugal
instability arises are extremely interesting since, although the necessary condition for
instability due to Rayleigh does successfully predict the spatial region within which
the instability develops, detailed analysis is required to determine the prefered axial
scale. A central motivation for the work to be described below is to understand
whether this mechanism (or, in fact, the family of mechanisms, as we will see) may
play an important role in circumstances that are geophysically interesting.''
    \item There is a eigenmode crossing at $d = 0.6$ in the L1 mode
    (non-rotating, unstratified KH wavetrain case). After $d=0.6$ the dominant
    instability appears to gather at the hyperbolic stagnation points of the KH
    billows and is therefore called the hyperbolic instability. They remark that
    this is insignificant for an isolated vortex tube as there would be no
    stagnation lines for this instability to gather onto. For the non-rotating
    stratified case, the instabtility becomes completely core-centered and only
    the L1 mode remains unstable. 

    \item In the rotating, unstratified case, there are three dominant
    instabtilities which appear in the $Ro^{-1}$ space. We have the L1 and L3
    modes, the Edge modes E0 and E1 (which is likely an essential part of
    anticyclone destruction), and the $\varepsilon$ mode which is highly
    core-centered (and only exists for elliptical vortices).
    \item The edge mode might be of use for my work. The edge mode is found
    where the Rayleigh criterion is satisfied. This is likely becaue the edge
    mode is related to the centrifugal instability and exists for anticyclones
    but not necessarily cyclones. 
    \item For the rotating stratified case, the dominant instability of the
    columnar vortices appears to be primarily controlled by the rossby number,
    though for each instabily, a decrease in the Froude number appears to
    lower the growth rate of the present instability and in some cases,
    completely stability the instability. 
    \end{itemize}

\colorbox{yellow}{Add a bibliography using natbib. Need to build a bib file for this}. 

\end{document}

\documentclass{article}

\usepackage{graphicx} % Required for inserting images
\usepackage[left=1in,right=1in,top=1in,bottom=1in]{geometry}
\usepackage{amsmath}
\usepackage{amsthm} %proof environment
\usepackage{amssymb}
\usepackage{amsfonts}
\usepackage{enumitem} %nice lists
\usepackage{verbatim} %useful for something 
\usepackage{xcolor}
\usepackage{setspace}
\usepackage{titlesec}
\usepackage{blindtext} % I have no idea what this is 
\usepackage{caption}  % need this for unnumbered captions/figures
\usepackage{natbib}
\usepackage{tikz}
\usepackage{hyperref}

\titleformat{\section}
{\bfseries\Large}
{Lecture \thesection:}
{5pt}{}

\begin{document}

\title{AM 260 - Computational Fluid Dynamics: Lecture Notes}
\author{Dante Buhl}
\doublespacing
\newcommand{\wrms}{w_{\text{rms}}}
\newcommand{\bs}[1]{\boldsymbol{#1}}
\newcommand{\tb}[1]{\textbf{#1}}
\newcommand{\bmp}[1]{\begin{minipage}{#1\textwidth}}
\newcommand{\emp}{\end{minipage}}
\newcommand{\R}{\mathbb{R}}
\newcommand{\C}{\mathbb{C}}
\newcommand{\N}{\mathcal{N}}
\newcommand{\K}{\bs{\mathrm{K}}}
\newcommand{\m}{\bs{\mu}_*}
\newcommand{\s}{\bs{\Sigma}_*}
\newcommand{\dt}{\Delta t}
\newcommand{\dx}{\Delta x}
\newcommand{\tr}[1]{\text{Tr}(#1)}
\newcommand{\Tr}[1]{\text{Tr}(#1)}
\newcommand{\Div}{\nabla \cdot}
\renewcommand{\div}{\nabla \cdot}
\newcommand{\Curl}{\nabla \times}
\newcommand{\Grad}{\nabla}
\newcommand{\grad}{\nabla}
\newcommand{\grads}{\nabla_s}
\newcommand{\gradf}{\nabla_f}
\newcommand{\xs}{x_s}
\newcommand{\xf}{x_f}
\newcommand{\x}{\bs{x}}
\newcommand{\ts}{t_s}
\newcommand{\tf}{t_f}
\newcommand{\pt}{\partial t}
\newcommand{\pz}{\partial z}
\newcommand{\uvec}{\bs{u}}
\newcommand{\nvec}{\hat{\bs{n}}}
\newcommand{\jvec}{\bs{j}}
\newcommand{\bvec}{\bs{B}}
\newcommand{\B}{\bs{B}}
\newcommand{\evec}{\bs{E}}
\newcommand{\E}{\bs{E}}
\newcommand{\vort}{\bs{\omega}}
\newcommand{\F}{\bs{F}}
\newcommand{\T}{\tilde{T}}
\newcommand{\ez}{\bs{e}_z}
\newcommand{\ex}{\bs{e}_x}
\newcommand{\ey}{\bs{e}_y}
\newcommand{\thetahat}{\hat{\bs{\theta}}}
\newcommand{\rhat}{\hat{\bs{r}}}
\newcommand{\zhat}{\hat{\bs{z}}}
\newcommand{\eo}{\bs{e}_{\bs{\Omega}}}
\newcommand{\ppt}[1]{\frac{\partial #1}{\partial t}}
\newcommand{\pptwo}[2]{\frac{\partial^2 #1}{\partial #2^2}}
\newcommand{\ppthree}[2]{\frac{\partial^3 #1}{\partial #2^3}}
\newcommand{\pp}[2]{\frac{\partial #1}{\partial #2}}
\newcommand{\ddt}[1]{\frac{d #1}{d t}}
\newcommand{\DDt}[1]{\frac{D #1}{D t}}
\newcommand{\DD}[2]{\frac{D #1}{D #2}}
\newcommand{\ppts}[1]{\frac{\partial #1}{\partial t_s}}
\newcommand{\pptf}[1]{\frac{\partial #1}{\partial t_f}}
\newcommand{\ppz}[1]{\frac{\partial #1}{\partial z}}
\newcommand{\ddz}[1]{\frac{d #1}{d z}}
\newcommand{\dd}[2]{\frac{d #1}{d #2}}
\newcommand{\ppzetas}[1]{\frac{\partial^2 #1}{\partial \zeta^2}}
\newcommand{\ppzs}[1]{\frac{\partial #1}{\partial z_s}}
\newcommand{\ppzf}[1]{\frac{\partial #1}{\partial z_f}}
\newcommand{\ppx}[1]{\frac{\partial #1}{\partial x}}
\newcommand{\ppxi}[1]{\frac{\partial #1}{\partial x_i}}
\newcommand{\ppxj}[1]{\frac{\partial #1}{\partial x_j}}
\newcommand{\ppy}[1]{\frac{\partial #1}{\partial y}}
\newcommand{\ppzeta}[1]{\frac{\partial #1}{\partial \zeta}}


\maketitle 

\tableofcontents

\setlength{\parindent}{0pt}
\setcounter{section}{1}

\pagebreak

\section{}

\subsection{Deriving Integral Quantities using FCV}

Using a finite control volume we can easily
obtain the following equality for the divergence of the flow field in a
compressible limit. 
\begin{gather*}
    \div{\uvec} = \frac{1}{\delta V} \DDt{\delta V}
\end{gather*}

Next we look at the continuity equation (density equation) using the FCV
(fixed in space). We consider some arbitrary FCV V with surface S. We say, the
net mass flow ``out'' of V through S is equal to the time rate of change
of the ``decrease'' of mass inside V. That is, 
\begin{gather*}
    -\int_S \rho\uvec \cdot \bs{dS} = \ppt{}\left( \int_V \rho dV\right)\\
    \int_V \ppt{\rho} + \div(\rho\uvec) dV = 0 
\end{gather*}
Note that here, the Reynolds transport theorem (adjustment to the density by
transport of the FCV) is not necessary due to the volume begin fixed in space. 
The same can be done with an FCV moving with the fluid although the derivation
initially gives the equations in non-conservative form (which is more of a
matter of notation than anything) 

\subsection{Deriving Quantities using IFE}
Looking at the fixed in space scenario using an infinitesimal fluid element. We consider a small
infinitesimal volume element and the inflow/outflow from that element. 
\begin{gather*}
    \text{Inflow: } \rho\uvec dydx\\
    \text{Outflow: } \rho\uvec + \grad(\rho\uvec)\cdot \bs{dV}\\
    \ppt{\rho} + \grad\cdot(\rho\uvec) = 0
\end{gather*}

Using an IFE moving with the fluid, we have that the equation becomes, 
\begin{gather*}
    \DDt{\rho dV} + \rho\div{\uvec} = 0 
\end{gather*}

\subsection{Hyperbolic evolution equations}

\begin{gather*}
    \ppt{u} + a\ppx{u} = 0
\end{gather*}




\end{document}

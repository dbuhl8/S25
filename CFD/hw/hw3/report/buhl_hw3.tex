\documentclass{article}

\usepackage{graphicx} % Required for inserting images
\usepackage[left=1in,right=1in,top=1in,bottom=1in]{geometry} \usepackage{amsmath}
\usepackage{amsthm} %proof environment
\usepackage{amsthm} %proof environment
\usepackage{amssymb}
\usepackage{amsfonts}
\usepackage{enumitem} %nice lists
\usepackage{verbatim} %useful for something 
\usepackage{xcolor}
\usepackage{setspace}
\usepackage{titlesec}
\usepackage{blindtext} % I have no idea what this is 
\usepackage{caption}  % need this for unnumbered captions/figures
\usepackage{natbib}
\usepackage{appendix}
\usepackage{tikz}
\usepackage{hyperref}

\titleformat{\section}{\bfseries\Large}{Problem \thesection:}{5pt}{}

\begin{document}

\title{AM 260 - Computational Fluid Dynamis: Homework 3}
\author{Dante Buhl}


\newcommand{\wrms}{w_{\text{rms}}}
\newcommand{\bs}[1]{\boldsymbol{#1}}
\newcommand{\tb}[1]{\textbf{#1}}
\newcommand{\bmp}[1]{\begin{minipage}{#1\textwidth}}
\newcommand{\emp}{\end{minipage}}
\newcommand{\R}{\mathbb{R}}
\newcommand{\C}{\mathbb{C}}
\newcommand{\N}{\mathcal{N}}
%\newcommand{\K}{\bs{\mathrm{K}}}
\newcommand{\m}{\bs{\mu}_*}
\newcommand{\s}{\bs{\Sigma}_*}
\newcommand{\dt}{\Delta t}
\newcommand{\dx}{\Delta x}
\newcommand{\tr}[1]{\text{Tr}(#1)}
\newcommand{\Tr}[1]{\text{Tr}(#1)}
\newcommand{\Div}{\nabla \cdot}
\renewcommand{\div}{\nabla \cdot}
\newcommand{\Curl}{\nabla \times}
\newcommand{\Grad}{\nabla}
\newcommand{\grad}{\nabla}
\newcommand{\grads}{\nabla_s}
\newcommand{\gradf}{\nabla_f}
\newcommand{\xs}{x_s}
\newcommand{\x}{\bs{x}}
\newcommand{\xf}{x_f}
\newcommand{\ts}{t_s}
\newcommand{\tf}{t_f}
\newcommand{\pt}{\partial t}
\newcommand{\pz}{\partial z}
\newcommand{\uvec}{\bs{u}}
\newcommand{\bvec}{\bs{B}}
\newcommand{\nvec}{\hat{\bs{n}}}
\newcommand{\tu}{\tilde{\uvec}}
\newcommand{\B}{\bs{B}}
\newcommand{\A}{\bs{A}}
\newcommand{\jvec}{\bs{j}}
\newcommand{\F}{\bs{F}}
\newcommand{\T}{\tilde{T}}
\newcommand{\ez}{\bs{e}_z}
\newcommand{\ex}{\bs{e}_x}
\newcommand{\ey}{\bs{e}_y}
\newcommand{\eo}{\bs{e}_{\bs{\Omega}}}
\newcommand{\ppt}[1]{\frac{\partial #1}{\partial t}}
\newcommand{\pp}[2]{\frac{\partial #1}{\partial #2}}
\newcommand{\pptwo}[2]{\frac{\partial^2 #1}{\partial #2^2}}
\newcommand{\ddtwo}[2]{\frac{d^2 #1}{d #2^2}}
\newcommand{\DDt}[1]{\frac{D #1}{D t}}
\newcommand{\ppts}[1]{\frac{\partial #1}{\partial t_s}}
\newcommand{\pptf}[1]{\frac{\partial #1}{\partial t_f}}
\newcommand{\ppz}[1]{\frac{\partial #1}{\partial z}}
\newcommand{\ddz}[1]{\frac{d #1}{d z}}
\newcommand{\ppzetas}[1]{\frac{\partial^2 #1}{\partial \zeta^2}}
\newcommand{\ppzs}[1]{\frac{\partial #1}{\partial z_s}}
\newcommand{\ppzf}[1]{\frac{\partial #1}{\partial z_f}}
\newcommand{\ppx}[1]{\frac{\partial #1}{\partial x}}
\newcommand{\ddx}[1]{\frac{d #1}{d x}}
\newcommand{\ppxi}[1]{\frac{\partial #1}{\partial x_i}}
\newcommand{\ppxj}[1]{\frac{\partial #1}{\partial x_j}}
\newcommand{\ppy}[1]{\frac{\partial #1}{\partial y}}
\newcommand{\ppzeta}[1]{\frac{\partial #1}{\partial \zeta}}
\renewcommand{\k}{\bs{k}}
\newcommand{\real}[1]{\text{Re}\left[#1\right]}


\maketitle 
% This line removes the automatic indentation on new paragraphs
\setlength{\parindent}{0pt}

\section{Lax-Friedrichs Method}


\section{Lax-Wendroff Method}

\begin{enumerate}[label = (\alph*)]
    \item Show that the LW method is convergent if $|C_a| \le 1$. 
    
        In order to demonstrate consistency and stability, we perform taylor
        expansions to demonstrate consistency (and at which order it is
        consistent), and then von Neumann stability analysis in order to prove
        stability.

        \textbf{Consistency}
        \begin{align*}
            U_j^{n+1} &= U_j^n + \Delta t U_{t, j}^n + 
            \frac{\Delta t^2}{2}U_{tt, j}^n + \frac{\Delta t^3}{6}U_{ttt, j}^n
            + \frac{\Delta t^4}{24}U_{tttt, j}^n + O(\Delta t^5)\\
            U_{j+1}^{n} &= U_j^n + \Delta x U_{x, j}^n + 
            \frac{\Delta x^2}{2}U_{xx, j}^n + \frac{\Delta x^3}{6}U_{xxx, j}^n
            + \frac{\Delta x^4}{24}U_{xxxx, j}^n + O(\Delta x^5)\\
            U_{j-1}^{n} &= U_j^n - \Delta x U_{x, j}^n + 
            \frac{\Delta x^2}{2}U_{xx, j}^n - \frac{\Delta x^3}{6}U_{xxx, j}^n
            + \frac{\Delta x^4}{24}U_{xxxx, j}^n + O(\Delta x^5)
        \end{align*}
        \begin{gather*}
            \lim_{\Delta t, \Delta x \to 0} E_{LT} = 
            \lim_{\Delta t, \Delta x \to 0} \frac{1}{\Delta t} U_j^{n+1} - U_j^n
            + \frac{1}{2}C_a\left(U_{j+1}^n - U_{j-1}^n\right)
            - \frac{1}{2}C_a^2\left(U_{j+1}^n - 2U_j^n + U_{j-1}^n\right)\\
           =  \lim_{\Delta t, \Delta x \to 0} U_{t, j}^n + 
            \frac{\Delta t}{2}U_{tt, j}^n + \frac{\Delta t^2}{6}U_{ttt, j}^n +
            O(\Delta t^3)
            + aU_{x, j}^n + a\frac{\Delta
            x^2}{6}U_{xxx, j}^n + O(\Delta x^4) \\- aC_a\left(
            \frac{\Delta x}{2}U_{xx, j}^n + \frac{\Delta x^3}{24}U_{xxxx, j}^n
            + O(\Delta x^5)\right)\\
            = \lim_{\Delta t, \Delta x \to 0} \frac{\Delta t^2}{6}U_{ttt, j}^n +
            a \frac{\Delta x^2}{6}U_{xxx, j}^n + O(\Delta^3)
        \end{gather*}
        Therefore, we have that this method is consistent with $O(\Delta t^2 +
        \Delta x^2)$. 

        \textbf{Stability} 
        \begin{gather*}
            G = (1 - C_a^2) + \frac{1}{2}\left(C_a^2 - C_a\right) e^{ik_x\Delta
            x} + \frac{1}{2}\left(C_a^2 + C_a\right) e^{-ik_x\Delta x}\\
            G = (1 - C_a^2) + C_a^2\cos(k_x \Delta x) - iC_a\sin(k_x \Delta x)\\
            |G| = (1 - C_a^2)^2 + C_a^4\cos^2(k_x \Delta x) + 2(1 -
            C_a^2)C_a^2\cos(k_x\Delta x) + C_a^2\sin^2(k_x\Delta x)\\
            = 1 - 2C_a^2 + C_a^4 + C_a^4\cos^2() + 2C_a^2\cos() - 2C_a^4\cos() +
            C_a^2\sin^2()\\
            = 1 + C_a^2\left(2\cos + \sin^2 - 2\right) + C_a^4\left(1 + \cos^2 -
            2\cos\right)
        \end{gather*}
        We proceed from here casewise. Take, $|C_a| = 1$. We have, 
        \begin{gather*}
            |G| = 1 + 2\cos - 2 + 1 - 2\cos + 1 = 1
        \end{gather*}
        in which case, the method is stable. 
        We next consider $|C_a| \le 1$.
        \begin{gather*}
            
        \end{gather*}

    

    \item Show that the LW method is $O(\Delta t^2 + \Delta x^2)$.
\end{enumerate}

\section{von Neumann Stability Analysis}
We can show that this method is unconditionally unstable with only a few lines
of algebra. 
\begin{gather*}
    U_j^{n+1} = U_j^n - \frac{a\Delta t}{2\Delta x}\left(U_{j+1}^n -
    U_{j-1}^n\right), \quad U_{j}^n = G^ne^{i j k_x \Delta x}\\
    G = 1 - \frac{a\Delta t}{2\Delta x}\left(e^{ik_x\Delta x}  - e^{-ik_x\Delta
    x}\right)\\
    G = 1 - \frac{a\Delta t}{2\Delta x}i\sin(k_x\Delta x)\\
    |G| = 1 + \left(\frac{a\Delta t}{2\Delta x}\right)^2\sin(k_x\Delta x)^2 > 1
\end{gather*}
Therefore, we have that this method is unconditionally unstable, i.e. there is
no condition on which $|G| \le 1$. 

\section{Modified Lax-Friedrichs Coefficient}

\section{von Neumann Analysis of the Heat Equation}

\section{Sinusoidal Adv. with LF}

\section{Discontinuous IC with LF}

\section{Sinusoidal Adv. with LW}

\section{Discontinuous IC with LW}


\end{document}
